\documentclass{article}
\usepackage[landscape]{geometry}
\usepackage{url}
\usepackage{multicol}
\usepackage{amsmath}
\usepackage{esint}
\usepackage{amsfonts}
\usepackage{tikz}
\usetikzlibrary{decorations.pathmorphing}
\usepackage{amsmath,amssymb}
\usepackage{pgfplots}
\usepackage{array}

\usepackage{colortbl}
\usepackage{xcolor}
\usepackage{mathtools}
\usepackage{amsmath,amssymb}
\usepackage{enumitem}
\makeatletter

\newcommand*\bigcdot{\mathpalette\bigcdot@{.5}}
\newcommand*\bigcdot@[2]{\mathbin{\vcenter{\hbox{\scalebox{#2}{$\m@th#1\bullet$}}}}}
\makeatother

\title{IB HL Math: Analysis and Approaches Formula Sheet}
\usepackage[brazilian]{babel}
\usepackage[utf8]{inputenc}
\usepackage{afterpage}

\advance\topmargin-.8in
\advance\textheight3in
\advance\textwidth3in
\advance\oddsidemargin-1.45in
\advance\evensidemargin-1.45in
\parindent0pt
\parskip2pt
\newcommand{\hr}{\centerline{\rule{3.5in}{1pt}}}

\begin{document}

\definecolor{background}{rgb}{1, 0.9, 0.8}
\pagecolor{background}\afterpage{\nopagecolor}


\begin{center}{}\huge\textbf{IB HL Math: Analysis and Approaches Formula Sheet}\\\LARGE{\textbf{Koral Kulacoglu}}\\

\end{center}
\begin{multicols*}{2}

\tikzstyle{mybox} = [draw=black, fill=white, very thick,
    rectangle, rounded corners, inner sep=10pt, inner ysep=10pt]
\tikzstyle{fancytitle} =[fill=black, text=white, font=\bfseries]


%------------ START ---------------

%------------ Exponent Laws ---------------
\begin{tikzpicture}
\node [mybox] (box){%
    \begin{minipage}{0.46\textwidth}
    
    \begin{itemize}
    \setlength\itemsep{0em}
        \item $ x^1 = x $
        \item $ x^0 = 1 $
        \item $ x^m \cdot x^n = x^{m+n} $
        \item $ \frac{x^m}{x^n} = x^{m-n} $
        \item $ (x^m)^n = m^{m \cdot n} $
        \item $ (x \cdot y)^n = x^n \cdot y^n $
        \item $ x^{-1} = \frac{1}{x} $
    \end{itemize}
    	
    \end{minipage}
};

%------------ Exponent Laws Header ---------------------
\node[fancytitle, right=10pt] at (box.north west) {Exponent Laws};
\end{tikzpicture}

%------------ Fractional Exponents ---------------
\begin{tikzpicture}
\node [mybox] (box){%
    \begin{minipage}{0.46\textwidth}
    
    \begin{itemize}
    \setlength\itemsep{0em}
        \item $ x^{\frac{1}{2}} = \sqrt{x} $
        \item $ \sqrt{x} \cdot \sqrt{x} = x $
        \item $ \sqrt{xy} = \sqrt{x} \cdot \sqrt{y} $
        \item $ x^{m/n} = \sqrt[n]{x^m} $

    \end{itemize}
    	
    \end{minipage}
};

%------------ Fractional Exponents Header ---------------------
\node[fancytitle, right=10pt] at (box.north west) {Fractional Exponents};
\end{tikzpicture}

%------------ Logarithmic Laws ---------------
\begin{tikzpicture}
\node [mybox] (box){%
    \begin{minipage}{0.46\textwidth}
    
    \begin{itemize}
    \setlength\itemsep{0em}
        \item $ \log_c{a} + \log_c{b} = log_c{(a \cdot b)}$
        \item $ \log_c{a} - \log_c{b} = log_c{(\frac{a}{b})}$
        \item $ n\log_c{a} = log_c{(a^n)}$
        \item $ \log_b{a} = \frac{log{a}}{log{b}}$

    \end{itemize}
    
    \quad\quad\quad\quad\quad $ \log_a{0} = x $ is undefined because $ a^x \neq 0 $
    
    \begin{tikzpicture}
    \begin{axis}[restrict y to domain=-10:10]
    \addplot [blue,domain=-10:10, samples=200]{log10(x)};
    \end{axis}
    \end{tikzpicture}
    \end{minipage}
};

%------------ Logarithmic Laws Header ---------------------
\node[fancytitle, right=10pt] at (box.north west) {Logarithmic Laws};
\end{tikzpicture}

%------------ Natural Logarithm ---------------
\begin{tikzpicture}
\node [mybox] (box){%
    \begin{minipage}{0.46\textwidth}
    
    The \textbf{natural logarithm} (ln) is log base e ($ log_e $). \\
    e is irrational: 2.71828...
    \begin{itemize}
    \setlength\itemsep{0em}
        \item $ \ln{a} + \ln{b} = \ln{(a \cdot b)} $
        \item $ \ln{a} - \ln{b} = \ln{(\frac{a}{b})} $
        \item $ n\ln{a} = \ln{(a^n)}$
        \item $ \ln{e} = 1 $
        \item $ e^{ln{(a)}} = a $

    \end{itemize}
    \end{minipage}
};

%------------ Natural Logarithm Header ---------------------
\node[fancytitle, right=10pt] at (box.north west) {Natural Logarithm};
\end{tikzpicture}



%------------ Common Difference ---------------
\begin{tikzpicture}
\node [mybox] (box){%
    \begin{minipage}{0.46\textwidth}
    
    The \textbf{common difference (d)} is the difference between two consecutive terms. \\
    
	\begin{tabular}{lp{8cm} l}
    	$ d = u_{n+1} - u_n $
    	& $ d $: common difference \\
    	& $ u $: sequence \\
    	& $ n $: term \\\\
	\end{tabular}
    
    Given $ \textbf{2},4,6,8... $ and $ n = 1 $ \\\\
	\begin{tabular}{lp{8cm} l}
    	$ d = u_{n+1} - u_n $ \\
    	$ d = 4 - 2 $ \\
    	$ d = \textbf{2} $
	\end{tabular}

    \end{minipage}
};

%------------ Common Difference Header ---------------------
\node[fancytitle, right=10pt] at (box.north west) {Common Difference - Arithmetic Sequences};
\end{tikzpicture}

%------------ N'th Term ---------------
\begin{tikzpicture}
\node [mybox] (box){%
    \begin{minipage}{0.46\textwidth}
    
    The n'th term can be found with the following formula: \\
    
	\begin{tabular}{lp{8cm} l}
    	$ u_n = u_1 + d(n-1) $
    	& $ d $: common difference \\
    	& $ u $: sequence \\
    	& $ n $: term \\\\
	\end{tabular}
    
    Given $ 2,4,\textbf{6},8... $ and $ n = 3 $ \\\\
	\begin{tabular}{lp{8cm} l}
    	$ u_n = u_1 + d(n-1) $ \\
    	$ u_3 = 2 + 2(2) $ \\
    	$ u_3 = \textbf{6} $
	\end{tabular}

    \end{minipage}
};

%------------ N'th Term Header ---------------------
\node[fancytitle, right=10pt] at (box.north west) {N'th Term - Arithmetic Sequences};
\end{tikzpicture}

%------------ Sum of N Terms ---------------
\begin{tikzpicture}
\node [mybox] (box){%
    \begin{minipage}{0.46\textwidth}
    
    The sum of n terms can be found with the following formula: \\
    
	\begin{tabular}{lp{8cm} l}
    	$ S_n = \frac{n}{2}(2u_1 + d(n-1)) $
    	& $ S $: sum of size n \\
    	& $ d $: common difference \\
    	& $ u $: sequence \\
    	& $ n $: term \\\\
	\end{tabular}
    
    Given $ \textbf{2,4,6,8}... $ and $ n = 4 $ \\\\
	\begin{tabular}{lp{8cm} l}
    	$ S_n = \frac{n}{2}(2u_1 + d(n-1)) $ \\
    	$ S_4 = \frac{4}{2}(2(2) + 2(4-1)) $ \\
    	$ S_4 = \textbf{20} $
	\end{tabular}

    \end{minipage}
};

%------------ Sum of N Terms Header ---------------------
\node[fancytitle, right=10pt] at (box.north west) {Sum of N Terms - Arithmetic Sequences};
\end{tikzpicture}

%------------ Finding the First Term & Common Difference ---------------
\begin{tikzpicture}
\node [mybox] (box){%
    \begin{minipage}{0.46\textwidth}
    
    $ u_{10} = 37 $ \\
    $ u_{22} = 1 $ \\
    Find the first term ($ u_1 $) and the common difference ($ d $).
    
    \begin{enumerate}
    \setlength\itemsep{0em}
        \item Place both numbers in the \textbf{n'th term formula}:\\
        $ u_n = u_1 + d(n-1) $ \\\\
        $ 37 = u_1 + 9d $ \\
        $ 1 = u_1 + 21d $ \\
        
        \item Equate them to find the \textbf{common difference}:\\
        $ 9d - 37 = 21d - 1 $ \\
        $ d = -3 $ \\

        \item Plug \textbf{d} into the n'th term formula to find the \textbf{first term}:\\
        $ 37 = u_1 + 9(-3) $
        $ u_1 = 64 $
    \end{enumerate}

    \end{minipage}
};

%------------ Finding the First Term & Common Difference term Header ---------------------
\node[fancytitle, right=10pt] at (box.north west) {Finding the First Term \& Common Difference - Arithmetic Sequences};
\end{tikzpicture}

%------------ N'th Term ---------------
\begin{tikzpicture}
\node [mybox] (box){%
    \begin{minipage}{0.46\textwidth}
    
    The n'th term can be found with the following formula: \\
    
	\begin{tabular}{lp{8cm} l}
    	$ u_n = u_1 \cdot r^{n-1} $
    	& $ r $: common ratio \\
    	& $ u $: sequence \\
    	& $ n $: term \\
	\end{tabular}
	
    \end{minipage}
};

%------------ N'th Term Header ---------------------
\node[fancytitle, right=10pt] at (box.north west) {N'th Term - Geometric Sequences};
\end{tikzpicture}

%------------ Sum of N Terms ---------------
\begin{tikzpicture}
\node [mybox] (box){%
    \begin{minipage}{0.46\textwidth}
    
    The sum of n terms can be found with the following formula: \\

	\begin{tabular}{lp{8cm} l}
    	$ S_n = \frac{u_1(1-r^n)}{1-r} $
    	& $ S $: sum of size n \\
    	& $ r $: common ratio \\
    	& $ u $: sequence \\
    	& $ n $: term \\
	\end{tabular}
	
    \end{minipage}
};

%------------ Sum of N Terms Header ---------------------
\node[fancytitle, right=10pt] at (box.north west) {Sum of N Terms - Geometric Sequences};
\end{tikzpicture}

%------------ Finding the First Term & Common Ratio - Geometric Sequences ---------------
\begin{tikzpicture}
\node [mybox] (box){%
    \begin{minipage}{0.46\textwidth}
    
    The sum of the first 5 terms is 3798. \\
    The sum to the end is 4374. \\
    Find the sum of the first 7 terms.
    
    \begin{enumerate}
    \setlength\itemsep{0em}
        \item Place the values into the \textbf{sum of n terms} \textbf{sum of all terms} formula:\\
        $ S_n = \frac{u_1(1-r^n)}{1-r} $ \\
        $ 3798 = \frac{u_1(1-r^5)}{1-r} $ \\
        
        $ S_n = \frac{u_1}{1-r} $ \\
        $ 4374 = \frac{u_1}{1-r} $ \\
        
        \item Rearrange and substitute $ u_1 $, then solve for the \textbf{common ratio} (r):\\
        $ u_1 = 4374(1-r) $ \\
        
        $ 3798 = \frac{4374(1-r)(1-r^5)}{1-r} $ \\
        $ r = \frac{2}{3} $ \\

        \item Plug the common ratio into the \textbf{sum of all terms} formula to find the \textbf{first term}:\\
        $ 4374 = \frac{u_1}{1-\frac{2}{3}} $ \\
        $ u_1 = 1458 $ \\
        
        \item Use the \textbf{sum of n terms} formula to find the sum of the first 7 terms:\\
        $ S_n = \frac{u_1(1-r^n)}{1-r} $ \\
        $ S_n = \frac{1458(1-(\frac{2}{3})^7)}{1-\frac{2}{3}} $ \\
        $ S_n = 4370 $

    \end{enumerate}

    \end{minipage}
};

%------------ Finding the First Term & Common Ratio - Geometric Sequences Header ---------------------
\node[fancytitle, right=10pt] at (box.north west) {Finding the First Term \& Common Ratio - Geometric Sequences};
\end{tikzpicture}

%------------ Sigma ---------------
\begin{tikzpicture}
\node [mybox] (box){%
    \begin{minipage}{0.46\textwidth}
    
    Sum the range:\\
	\LARGE{$ \sum_{n=start}^{end} formula $}\\\\
	\normalsize{$ \sum_{n=1}^{3} 2n = 2(1) + 2(2) + 2(3) = 12$}
    	
    \end{minipage}
};

%------------ Sigma Header ---------------------
\node[fancytitle, right=10pt] at (box.north west) {Sigma};
\end{tikzpicture}

%------------ Combinations Formula ---------------
\begin{tikzpicture}
\node [mybox] (box){%
    \begin{minipage}{0.46\textwidth}

	\begin{tabular}{lp{8cm} l}
    	\LARGE{$ C_k^n = {n \choose k} = \frac{n!}{k!(n-k)!} $} \\
    	& $ C_r^n $: number of combinations \\
    	& $ n $: total number of objects in the set \\
    	& $ k $: number of choosing objects from the set \\\\
	\end{tabular}
	
	How many ways can you arrange 'k' from a set of 'n' if the order does not matter?

    \end{minipage}
};

%------------ Combinations Formula Header ---------------------
\node[fancytitle, right=10pt] at (box.north west) {Combinations Formula};
\end{tikzpicture}

%------------ Permutations Formula ---------------
\begin{tikzpicture}
\node [mybox] (box){%
    \begin{minipage}{0.46\textwidth}

	\begin{tabular}{lp{8cm} l}
    	\LARGE{$ P_k^n = \frac{n!}{(n-k)!} $} \\
    	& $ P_n^k $: number of permutations \\
    	& $ n $: total number of objects in the set \\
    	& $ k $: number of choosing objects from the set \\\\
	\end{tabular}
	
	How many ways can you arrange 'k' from a set of 'n' if the order matters?

    \end{minipage}
};

%------------ Permutations Formula Header ---------------------
\node[fancytitle, right=10pt] at (box.north west) {Permutations Formula};
\end{tikzpicture}

%------------ Binomial Expansion ---------------
\begin{tikzpicture}
\node [mybox] (box){%
    \begin{minipage}{0.46\textwidth}

    General formula: \\
	\LARGE{$ (a+b)^n = \sum_{k=0}^{n} {n \choose k} a^{n-k} b^k $} \\

    \large
	\textbf{a}'s exponent \textbf{decreases}. \\
	\textbf{b}'s exponent \textbf{increases}.

    \begin{itemize}
    \setlength\itemsep{0em}
        \item $ (a+b)^2 = a^2 + 2ab + b^2 $
        \item $ (a+b)^3 = a^3 + 3a^2b + 3ab^2 + b^3 $
        \item $ (a+b)^n = {n \choose 0}a^{n-0}b^0 + {n \choose 1}a^{n-1}b^1 + {n \choose 2}a^{n-2}b^2... $ \\
    \end{itemize}
    
    \normalsize
    Finding the coefficient of a specific term. \\
    
    expansion: $ (2x - 5)^8 $ \\
    term: $ x^5 $
    
    \begin{enumerate}
        \setlength\itemsep{0em}
        \item Find k. \\
        Since x is the first term, k is \textbf{$ 8 - 5 $}. \\ (\textbf{expansion's exponent} - \textbf{the term's exponent}):\\\\
        $ k = n - 5 $ \\
        $ k = 8 - 5 = 3 $ \\
        
        \item Plug n and k into the \textbf{general formula}:\\
        (a and b are the two terms 2x and -5)

        \LARGE
    	Formula: $ {n \choose k}a^{n-k}b^k $
    	
    	\large
    	$ = {8 \choose 3}(2x)^{8-3}(-5)^3 $ \\
    	$ = 56(2x)^{5}(-5)^3 $ \\
    	$ = -224000x^5 $
    \end{enumerate}

    \end{minipage}
};

%------------ Binomial Expansion Header ---------------------
\node[fancytitle, right=10pt] at (box.north west) {Binomial Expansion};
\end{tikzpicture}

%------------ Pascal's Triangle ---------------
\begin{tikzpicture}
\node [mybox] (box){%
    \begin{minipage}{0.46\textwidth}
    
    \begin{tabular}{>{n=}l<{\hspace{12pt}}*{13}{c}}
    0 &&&&&&&1&&&&&&\\
    1 &&&&&&1&&1&&&&&\\
    2 &&&&&1&&2&&1&&&&\\
    3 &&&&1&&3&&3&&1&&&\\
    4 &&&1&&4&&6&&4&&1&&\\
    5 &&1&&5&&10&&10&&5&&1&\\
    6 &1&&6&&15&&20&&15&&6&&1
    \end{tabular}

    \end{minipage}
};

%------------ Pascal's Triangle Header ---------------------
\node[fancytitle, right=10pt] at (box.north west) {Pascal's Triangle};
\end{tikzpicture}

%------------ Induction ---------------
\begin{tikzpicture}
\node [mybox] (box){%
    \begin{minipage}{0.46\textwidth}

    Steps :
    \begin{enumerate}
        \setlength\itemsep{0em}
        \item Showing the statement holds for the first case, $ n = 1 $.
        \item Assuming the statement holds true for some value, $ n = k $.
        \item Proving, using the assumption in 2, that the statement holds for $ n = k + 1 $.
    \end{enumerate}

    \end{minipage}
};

%------------ Induction Header ---------------------
\node[fancytitle, right=10pt] at (box.north west) {Induction};
\end{tikzpicture}

%------------ Complex Number Forms ---------------
\begin{tikzpicture}
\node [mybox] (box){%
    \begin{minipage}{0.46\textwidth}

    Forms of complex numbers (z)... \\
    
    Cartesian: \\
	\begin{tabular}{lp{8cm} l}
        $ z = a + bi $
        & $ i = \sqrt{-1} $ \\
        & $ a \in \mathbb{R} $ \\
        & $ b \in \mathbb{C} $ \\
	\end{tabular}

    Polar: \\
	\begin{tabular}{lp{8cm} l}
        $ z = r(\cos{\theta} + i\sin{\theta}) $
        & r: modulus (distance from origin) \\
        & $ \theta $: argument (angle from x-axis to point) \\\\
	\end{tabular}
	
	Polar coordinates use the distance and angle. \\
	
        \begin{tikzpicture}
            \draw (0,0) grid (3,3);
            \fill[blue] (1,2) circle (2pt);
            \draw[blue] (0:0) -- +(1,2);
        \end{tikzpicture}
    
    $ \theta = \arctan({2/1}) = 63^{\circ} $ (adj/opp) \\
    $ r = \sqrt{1^2+2^2} = \sqrt{5} $ (Pythagorean Theorem) \\

    \end{minipage}
};

%------------ Complex Number Forms Header ---------------------
\node[fancytitle, right=10pt] at (box.north west) {Complex Number Forms};
\end{tikzpicture}

%------------ Complex Number Conjugate ---------------
\begin{tikzpicture}
\node [mybox] (box){%
    \begin{minipage}{0.46\textwidth}

    The \textbf{conjugate} of a complex number: \\
    $ \bar{z} = a - bi $
    
    \end{minipage}
};

%------------ Complex Number Conjugate Header ---------------------
\node[fancytitle, right=10pt] at (box.north west) {Complex Number Conjugate};
\end{tikzpicture}

%------------ Complex Number Operations - Cartesian ---------------
\begin{tikzpicture}
\node [mybox] (box){%
    \begin{minipage}{0.46\textwidth}

    When dividing, since a \textbf{complex number multiplied by its conjugate} is a \textbf{real number} they can be rationalized. \\

    For example: Rewrite $ \frac{2+6i}{1-2i} $ in the form: $ a + b1 $. \\
    
    $ = \frac{(2+6i)(1+2i)}{(1-2i)(1+2i)} $ \\
    
    $ = \frac{-10+10i}{5} = -2+2i $ \\

    \LARGE
    Remember \textbf{$ i^2 = -1 $}

    \end{minipage}
};

%------------ Complex Number Operations - Cartesian Header ---------------------
\node[fancytitle, right=10pt] at (box.north west) {Complex Number Operations - Cartesian};
\end{tikzpicture}

%------------ Complex Number Operations - Polar ---------------
\begin{tikzpicture}
\node [mybox] (box){%
    \begin{minipage}{0.46\textwidth}

    Polar form: \\
    \LARGE
    $ cis(x) = \cos{x} + i\sin{x} $ \\
    
    \normalsize
    Given $ z_1 = r_1 cis(\theta_1) $ and $ z_2 = r_2 cis(\theta_2) $: \\
    
    Multiplication: \\
    $ z_1 \cdot z_2 = r_1 \cdot r_2 \cdot cis(\theta_1+\theta_2)$ \\
    
    Multiplication: \\
    $ \frac{z_1}{z_2} = \frac{r1}{r2} cis(\theta_1-\theta_2)$

    \end{minipage}
};

%------------ Complex Number Operations - Polar Header ---------------------
\node[fancytitle, right=10pt] at (box.north west) {Complex Number Operations - Polar};
\end{tikzpicture}

%------------ Euler's Theorem ---------------
\begin{tikzpicture}
\node [mybox] (box){%
    \begin{minipage}{0.46\textwidth}
    
    \LARGE
    $ e^{ix} = \cos{x} + i\sin{x} $

    \end{minipage}
};

%------------ Euler's Theorem ---------------------
\node[fancytitle, right=10pt] at (box.north west) {Euler's Theorem};
\end{tikzpicture}

%------------ De Moivre's Theorem ---------------
\begin{tikzpicture}
\node [mybox] (box){%
    \begin{minipage}{0.46\textwidth}
    
    \LARGE
    $ z^n = (\cos{x} + i\sin{x})^n = \cos{(nx)} + i\sin{(nx)} $

    \end{minipage}
};

%------------ De Moivre's Theorem ---------------------
\node[fancytitle, right=10pt] at (box.north west) {De Moivre's Theorem};
\end{tikzpicture}

%------------ N'th Roots of Complex Numbers ---------------
\begin{tikzpicture}
\node [mybox] (box){%
    \begin{minipage}{0.46\textwidth}
    
    The n't root of a complex number in the polar form ($ z = r cis(\theta) $): \\
    \LARGE
    $ z_{k+1} = \sqrt[n]{r} cis(\frac{\theta}{n} + \frac{2k\pi}{n}) $ \\

    \normalsize
    
    Find 3 roots of $ z^3 = 4 + 4\sqrt{3i} $ \\
    
    \begin{enumerate}
    \setlength\itemsep{0em}
        \item Convert to polar form: \\
        $ r = \sqrt{4^2 + (4\sqrt{3})^2} = 8 $ \\
        $ \theta = \arctan{\sqrt{3}} = \frac{\pi}{3} $ \\\\
        $ z^3 = 8cis(\frac{pi}{3}) $ \\
        
        \item Plug values into the \textbf{n'th term formula}:\\
        \LARGE
        $ z_1 = \sqrt[3]{8}cis(\frac{\pi}{3 \cdot 3} + \frac{2 \cdot 0 \cdot \pi}{3}) = 2cis(\frac{\pi}{9}) $ \\
        $ z_2 = \sqrt[3]{8}cis(\frac{\pi}{3 \cdot 3} + \frac{2 \cdot 1 \cdot \pi}{3}) = 2cis(\frac{7\pi}{9}) $ \\
        $ z_3 = \sqrt[3]{8}cis(\frac{\pi}{3 \cdot 3} + \frac{2 \cdot 2 \cdot \pi}{3}) = 2cis(\frac{13\pi}{9}) $
    \end{enumerate}

    \end{minipage}
};

%------------ N'th Roots of Complex Numbers ---------------------
\node[fancytitle, right=10pt] at (box.north west) {N'th Roots of Complex Numbers};
\end{tikzpicture}

%------------ Linear Functions ---------------
\begin{tikzpicture}
\node [mybox] (box){%
    \begin{minipage}{0.46\textwidth}

    \LARGE
    $ f(x) = mx + b $
    
    \normalsize
    \begin{itemize}
    \setlength\itemsep{0em}
        \item Midpoint: $ \frac{x_1+x_2}{2}, \frac{y_1+y_2}{2} $
        \item Distance: $ \sqrt{(x_2-x_1)^2 + (y_2-y_1)^2} $
        \item Slope: $ \frac{y_2-y_1}{x_2-x_1} $
        \item Parallel: $ m_1 = m_2 $ (same slope)
        \item Perpendicular: $ m_1 = -\frac{1}{m_2}$ (negative reciprocal)
    \end{itemize}

    \end{minipage}
};

%------------ Linear Functions ---------------------
\node[fancytitle, right=10pt] at (box.north west) {Linear Functions};
\end{tikzpicture}

%------------ Quadratic Functions ---------------
\begin{tikzpicture}
\node [mybox] (box){%
    \begin{minipage}{0.46\textwidth}

    Quadratic Forms: \\
    Standard: \LARGE $ f(x) = ax^2 + bx + c $ \normalsize \\
    Factorized: \LARGE $ f(x) = a(x-h)^2 + k $ \normalsize \\
    Vertex: \LARGE $ f(x) = a(x-p)(x-q) $ \normalsize \\
    
    \begin{itemize}
    \setlength\itemsep{0em}
        \item Roots/X-intercepts: where $ f(x) = 0 $
        \item Quadratic Formula: $ \frac{-b +- \sqrt{b^2-4ac}}{2a} $
        \item Discriminant: $ \sqrt{b^2-4ac} $ (2nd parameter of the quadratic formula)
        \item Parallel: $ m_1 = m_2 $ (same slope)
        \item Perpendicular: $ m_1 = -\frac{1}{m_2}$ (negative reciprocal) \\
    \end{itemize}
    
    When the discriminant $ \Delta $ is:
    \begin{itemize}
    \setlength\itemsep{0em}
        \item $ \Delta > 0 \Rightarrow $ 2 solutions
        \item $ \Delta = 0 \Rightarrow $ 1 solution
        \item $ \Delta < 0 \Rightarrow $ 0 solutions \\
    \end{itemize}
        
    Completing the square (vertex form): \\\\
    $ f(x) = 2x^2 - 16x + 10 $
    \begin{enumerate}
    \setlength\itemsep{0em}
        \item Factor out a (coefficient of $ x^2 $):\\
        $ = 2(x^2-8x)+10 $ \\

        \item Add and subtract the halved and squared version of the coefficient of x: \\
        $ = 2(x^2-8x+16-16) + 10 $ \\

        \item Bring the subtracted term outside: \\
        $ = 2(x^2-8x+16)-32+10 $ \\
        
        \item The inner terms are a perfect square. Factor them: \\
        $ = 2(x-4)^2-22 $ \\
    \end{enumerate}

    \end{minipage}
};

%------------ Quadratic Functions ---------------------
\node[fancytitle, right=10pt] at (box.north west) {Quadratic Functions};
\end{tikzpicture}

%------------ Exponential Functions ---------------
\begin{tikzpicture}
\node [mybox] (box){%
    \begin{minipage}{0.46\textwidth}

    \LARGE
    $ f(x) = a^{x+b}+c $ \\

    \normalsize
    If  $ c >= 0 $: 0 roots

    \end{minipage}
};

%------------ Exponential Functions ---------------------
\node[fancytitle, right=10pt] at (box.north west) {Exponential Functions};
\end{tikzpicture}


%------------ Rational Functions ---------------
\begin{tikzpicture}
\node [mybox] (box){%
    \begin{minipage}{0.46\textwidth}

    \LARGE
    $ f(x) = \frac{ax+b}{cx+b} $ \\
    
    \normalsize
    Horizontal Asymptote: \\
    $\lim_{x \to \infty} f(x) = \frac{a}{c} $ \\
    
    Vertical Asymptote: \\
    Occurs when the denominator is equal to 0 $ \therefore x = -\frac{d}{c} $

    \end{minipage}
};

%------------ Rational Functions ---------------------
\node[fancytitle, right=10pt] at (box.north west) {Rational Functions};
\end{tikzpicture}

%------------ Logarithmic Functions ---------------
\begin{tikzpicture}
\node [mybox] (box){%
    \begin{minipage}{0.46\textwidth}

    \LARGE
    $ f(x) = \log_a{x} $ \\
    
    \normalsize
    Inverse exponential function. \\
    Has one root at $ x = 1 $.

    \end{minipage}
};

%------------ Logarithmic Functions ---------------------
\node[fancytitle, right=10pt] at (box.north west) {Logarithmic Functions};
\end{tikzpicture}

%------------ Even Functions ---------------
\begin{tikzpicture}
\node [mybox] (box){%
    \begin{minipage}{0.46\textwidth}

    \LARGE
    $ f(x) = f(-x) $
    
    \end{minipage}
};

%------------ Even Functions ---------------------
\node[fancytitle, right=10pt] at (box.north west) {Even Functions};
\end{tikzpicture}

%------------ Odd Functions ---------------
\begin{tikzpicture}
\node [mybox] (box){%
    \begin{minipage}{0.46\textwidth}

    \LARGE
    $ -f(x) = f(-x) $
    
    \end{minipage}
};

%------------ Odd Functions ---------------------
\node[fancytitle, right=10pt] at (box.north west) {Odd Functions};
\end{tikzpicture}

%------------ Inverse Functions ---------------
\begin{tikzpicture}
\node [mybox] (box){%
    \begin{minipage}{0.46\textwidth}

    Inverse: $ f^{-1}(x) $ \\
    To invert a function, $ y = x $ becomes $ x = y $.
    
    \end{minipage}
};

%------------ Inverse Functions ---------------------
\node[fancytitle, right=10pt] at (box.north west) {Inverse Functions};
\end{tikzpicture}

%------------ Function Transformations ---------------
\begin{tikzpicture}
\node [mybox] (box){%
    \begin{minipage}{0.46\textwidth}

    \begin{tabular}{l|l}
        \textbf{Change} & \textbf{Effect} \\ \hline
    	$ a \cdot f(x) $ & Vertical stretch by a \\ \hline
    	$ f(a \cdot x) $ & Horizontal stretch by 1/a \\ \hline
    	$ - f(x) $ & Vertical reflection over x-axis \\ \hline
    	$ f(-x) $ & Horizontal reflection over y-axis \\ \hline
    	$ f(x) + a $ & Upwards shift by a \\ \hline
    	$ f(x+a) $ & Left shift by a \\ \hline
    \end{tabular}

    \end{minipage}
};

%------------ Function Transformations ---------------------
\node[fancytitle, right=10pt] at (box.north west) {Function Transformations};
\end{tikzpicture}

%------------ Derivatives ---------------
\begin{tikzpicture}
\node [mybox] (box){%
    \begin{minipage}{0.46\textwidth}

    Polynomials: \\
    $ f(x) = ax^n \\ $
    $ f'(x) = nax^{n-1} $ \\
    (bring down the exponent) \\
    
    Constants: \\
    $ f(x) = a $ \\
    $ f'(x) = 0 $ \\
    (always 0) \\

    Sums: \\
    $ f(x) = ax^n, g(x) = bx^m $ \\
    $ f'(x) + g'(x) = nax^{n-1} + max^{m-1} $ \\
    (sum of both derivatives)

    \end{minipage}
};

%------------ Derivatives ---------------------
\node[fancytitle, right=10pt] at (box.north west) {Derivatives};
\end{tikzpicture}

%------------ Product Rule - Derivatives ---------------
\begin{tikzpicture}
\node [mybox] (box){%
    \begin{minipage}{0.46\textwidth}
    
    \Large
    When functions are multiplied $ y = uv $: \\
    Then: $ y' = uv' + u'v $ \\
    Same as: $ \frac{dy}{dx} = u \frac{dv}{dx} + v \frac{du}{dx} $

    \end{minipage}
};

%------------ Product Rule - Derivatives ---------------------
\node[fancytitle, right=10pt] at (box.north west) {Product Rule - Derivatives};
\end{tikzpicture}

%------------ Quotient Rule - Derivatives ---------------
\begin{tikzpicture}
\node [mybox] (box){%
    \begin{minipage}{0.46\textwidth}

    \Large
    When functions are divided $ y = \frac{u}{v} $: \\
    Then: $ y' = \frac{u'v-uv'}{v^2} $ \\
    Same as: $ \frac{dy}{dx} = \frac{v \frac{du}{dx} - u \frac{dv}{dx}}{v^2} $

    \end{minipage}
};

%------------ Quotient Rule - Derivatives ---------------------
\node[fancytitle, right=10pt] at (box.north west) {Quotient Rule - Derivatives};
\end{tikzpicture}

%------------ Chain Rule - Derivatives ---------------
\begin{tikzpicture}
\node [mybox] (box){%
    \begin{minipage}{0.46\textwidth}

    \Large
    Function in another function: $ y = g(u) $ where $ u = f(x) $. \\
    Then: $ \frac{dy}{dx} = \frac{dy}{du} \cdot \frac{du}{dx} $

    \end{minipage}
};

%------------ Chain Rule - Derivatives ---------------------
\node[fancytitle, right=10pt] at (box.north west) {Chain Rule - Derivatives};
\end{tikzpicture}

%------------ Common Derivatives ---------------
\begin{tikzpicture}
\node [mybox] (box){%
    \begin{minipage}{0.46\textwidth}

    \begin{tabular}{l|l|l}
        \textbf{Trigonometric} \\ \hline
    	$ \sin{x}' = \cos{x} $ & $ \arcsin{x}' = \frac{1}{\sqrt{1-x^2}} $ & $ \sec{x}' = \sec{x} \cdot \tan{x} $ \\ \hline
    	$ \cos{x}' = -\sin{x} $ & $ \arccos{x}' = -\frac{1}{\sqrt{1-x^2}} $ & $ \csc{x}' = -\csc{x} \cdot \cot{x} $ \\ \hline
    	$ \tan{x}' = \sec^2{x} $ & $ \arctan{x}' = \frac{1}{1+x^2} $ & $ \cot{x}' = -\csc^2{x} $ \\ \hline
    \end{tabular}

    \phantom \\\\    

    \begin{tabular}{l|l}
        \textbf{Exponential} & \textbf{Logarithmic} \\ \hline
    	$ (e^x)' = e^x $ & $ (ln(x))' = \frac{1}{x} $ \\ \hline
    	$ (a^x)' = a^x(\ln{a}) $ & $ (log_a{x})' = \frac{1}{x \ln{a}} $ \\ \hline
    \end{tabular}

    \end{minipage}
};

%------------ Common Derivatives ---------------------
\node[fancytitle, right=10pt] at (box.north west) {Common Derivatives};
\end{tikzpicture}


\end{multicols*}
\end{document}

